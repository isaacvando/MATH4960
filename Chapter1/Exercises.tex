\documentclass[12pt]{article}
\usepackage{amsmath}
\usepackage{amsfonts}
\usepackage{amssymb}
\usepackage{mathtools}
\usepackage{parskip}
\usepackage{fancyhdr}
\usepackage{tikz-cd}
\usepackage{geometry}
\usepackage[normalem]{ulem}
\newcommand\hl{\bgroup\markoverwith
    {\textcolor{yellow}{\rule[-.5ex]{.1pt}{2.5ex}}}\ULon}

\geometry{
  a4paper,
  left= 1in,
  right= 1in,
  top=1.42in,
}
\pagestyle{fancy}

\renewcommand{\headrulewidth}{0.5pt}
\renewcommand{\footrulewidth}{0.5pt}
\newcommand{\N}{\mathbb{N}}
\newcommand{\Z}{\mathbb{Z}}
\newcommand{\Q}{\mathbb{Q}}
\newcommand{\R}{\mathbb{R}}
\newcommand{\C}{\mathbb{C}}
\newcommand{\F}{\mathbb{F}}
\newcommand{\ER}{\mathcal{R}}
\newcommand{\im}{\text{im}}
\newcommand{\id}{\text{id}}
\renewcommand{\char}{\text{char}}
\newcommand{\coker}{\text{coker}}
\newcommand{\rank}{\text{rank}}
\newcommand{\row}{\text{row}}
\newcommand{\col}{\text{col}}
\newcommand{\nul}{\text{nul}}
\newcommand{\Tr}{\text{Tr}}
\newcommand{\exercise}[1]{\fbox{\textbf{Exercise #1}}}
\newcommand{\bitem}[1]{\item[\textbf{#1}]}
\newcommand{\nxn}{M_{n \times n}(K)}
\newcommand{\adj}{\text{adj}}
\newcommand{\Mult}{\text{Mult}_K(V^m, W)}
\newcommand{\Alt}{\text{Alt}_K(V^m, W)}
\newcommand{\ra}{\rightarrow}
\newcommand{\sgn}{\text{sgn}}
\newcommand{\amult}{\text{a-Mult}}
\newcommand{\gmult}{\text{g-Mult}}
\renewcommand{\null}{\text{null}}
\newcommand{\Min}{\text{Min}}
\newcommand{\End}{\text{End}}
\newcommand{\tors}{M_{\text{tors}}}
\newcommand{\ann}[2][R]{\text{Ann}_{#1}(#2)}
\newcommand{\kx}{K[x]}
\newcommand{\cat}[1]{\upshape{\sffamily{#1}}}
\newcommand{\commaArrow}{\!\downarrow\!}
\newcommand{\oper}{\operatorname}
\newcommand{\op}{\mathrm{op}}
\newcommand{\obj}{\mathrm{obj}}
\newcommand{\mor}{\mathrm{mor}}

\newcommand\restr[2]{{% we make the whole thing an ordinary symbol
\left.\kern-\nulldelimiterspace % automatically resize the bar with \right
#1 % the function
% \littletaller % pretend it's a little taller at normal size
\right|_{#2} % this is the delimiter
}}
\newcommand{\littletaller}{\mathchoice{\vphantom{\big|}}{}{}{}}

\lhead{Isaac Van Doren}
\chead{Chapter 1}
\rhead{\today}
\lfoot{}
\rfoot{Category Theory}

\begin{document}
  \subsection*{Section 1}
  \exercise{1.1.i} 
  \begin{itemize}
    \item[(i)] Let $f : x \ra y$ be a morphism and $g, g' : y \ra x$ be inverses. Then $fg = \id_y$ and $gf = \id_x$ (resp $g'$). Hence $gf = g'f$ so 
    \begin{align*} 
      gfg &= g'fg \\
      g &= g'
    \end{align*}
  Therefore any morphism has at most one inverse. ///

  \item[(ii)] Suppose there are morphisms $g,h : y \rightrightarrows x$ such that $fg = \id_y$ and $hf = \id_x$. Then
    \begin{align*}
      hfg &= h \\
      g &= h
    \end{align*}
  Hence $fg = \id_y$ and $gf = \id_x$ so $f$ is an isomorphism. ///
  \end{itemize}

  \exercise{1.1.ii}
  Consider a category $C$ and the subset of isomorphisms it contains. The subset contains the identity morphisms for each object because the identity is an isomorphism. The subset inherits associativity from $C$. Now pick two isomorphisms $f : x \ra y$ and $g : y \ra z$ and consider $gf$. We have $(f^{-1}g^{-1})gf = f^{-1}g^{-1}gf = f^{-1}f = \id_x$ and $gf(f^{-1}g^{-1}) = \id_y$ so $gf$ is also an isomorphism. Hence $C$ contains a maximal groupoid. ///
  

  \exercise{1.1.iii}
  \begin{itemize}
    \item[(i)] Observe that for any object $f : c \ra x$ in $c/C$, the morphism $\id_x : x \ra x$ is the identity morphism for $f$. Now pick two morphisms in $c/C$, $h : x \ra y$ and $k : y \ra z$ and consider $kh$.
  \begin{center}
    \begin{tikzcd}
      & c \arrow[ld, "f"'] \arrow[d, "g" description] \arrow[rd, "b"] &   \\
      x \arrow[r, "h"] & y \arrow[r, "k"] & z
    \end{tikzcd}
  \end{center}

  Because $hf = g$ and $kg = b$ we have $khf = b$ so $kh$ is a morphism in $c/C$ which is enough to show it is a category. ///

    \item[(ii)] A similar argument as above shows that $C/c$ is a category. ///
  \end{itemize}


  \subsection*{Section 2}
  % \exercise{1.2.ii}
  % \begin{itemize}
  %   \item[(i)] $(\implies)$ Suppose $f : x \ra y$ is a split epimorphism. Now pick a morphism $h$ in $C(c,y)$. We may write 
  % \end{itemize}


  \exercise{1.2.iii}
  Let $f : x \hookrightarrow y$ and $g : y \hookrightarrow z$ be monomorphisms and $h,k : w \ra x$ be morphisms such that $gfh = gfk$. Because $g$ is monic, it follows that $fh = fk$. Similarly, because $f$ is monic we have $h = k$. Hence $gf$ is monic. 

  Now suppose $f$ and $g$ are simply morphisms and that $gf$ is monic. BWOC suppose that $f$ is not monic. Then there exist morphisms $h,k$ such that $fh = fk$ but $h \ne k$. Hence we have $gfh = gfk$ this is a contradiction because $gf$ being monic implies that $h = k$. Hence $f$ is monic. 

  Now observe that the monomorphisms in a category form a subcategory. Every object has an identity morphism because the identity is monic. Associativity is inherited from the larger category and composition is preserved because the composition of two monomorphisms is also a monomorphisms as shown above.

  By duality the epimorphisms also form a subcategory. ///

  \exercise{1.2.iv} Recall that the only ideals in a field $F$ are $\{0\}$ and $F$. Pick a morphism $f : X \ra Y$ in \cat{Field}. The kernel of $f$ is an ideal so it is either $\{0\}$ or $X$. By the first isomorphism theorem, $f$ is either 1-1 or the 0 map. Now pick $g,h$ such that $fg = fh$. Then because $f$ is 1-1, then it maps every element of it's domain to a unique element in the codomain hence $h = g$. There for $f$ is monic. Hence the monomorphisms in \cat{Field} are the non-zero maps. ///

  *\exercise{1.2.v} In \cat{Ring}, consider the inclusion map $i : \Z \hookrightarrow \Q$. Now pick two morphisms, $f,g : R \rightrightarrows \Z$ such that $if = ig$ where $R$ is some ring. Now BWOC suppose that $f \ne g$. Then there exists an $r \in R$ such that $f(r) \ne g(r)$. Hence $i(f(r)) \ne i(g(r))$ so $if \ne ig$ which is a contradiction. Hence the above implies that $f = g$ so $i$ is monic. 

  Now pick morphisms $h,k : \Q \rightrightarrows R$ such that $hi = ki$. This means that for all integers $z$, $h(z) = k(z)$. Each rational $q$ may be written canonically as $nd^{-1}$ for $n, d \in \Z$. Hence $h(q) = h(nd^{-1}) = h(n)h(d^{-1})$. Further, because every element in $\Q$ is a unit, $h(n)h(d^{-1}) = h(n)h(d)^{-1}$. Because $h$ and $k$ agree on the integers, we have $h(n)h(d)^{-1} = k(n)k(d)^{-1}$. Hence $h = k$ and $i$ is an epimorphism. 

  $\Q$ and $\Z$ are not isomorphic in \cat{Ring} because $\Q$ is a field while $\Z$ is not. Hence in \cat{Ring}, monic and epic does not imply isomorphism. /// 

  \exercise{1.2.vii} For a given subset of objects $A \subseteq P$, consider $M_A = \{ x \in \mathrm{obj}P | \forall a \in A, \exists f \in P(a,x) \}$, i.e. the set of all elements in $P$ targeted by a morphism from every element in $A$. If it exists, the supremum of $A$ is the least element in $M_A$ w.r.t. $\le$. 

  We define the infimum by duality as the greatest element in $m_A = \{ x \in \mathrm{obj}P | \forall a \in A, \exists f \in P(x,a) \}$.

  Now suppose we have $p,q$ both suprema of $A$. Then by definition $p \le q$ and $q \le p$ because each is a member of $A$. Hence by the antireflexivity of $\le$, we have $p = q$. The dual proof shows uniquens for the infimum. ///



  \subsection*{Section 3}
  \exercise{1.3.i} Given two groups $G$ and $H$, a functor $\mathrm{B}G \ra \mathrm{B}H$ is a group homomorphism. Given such a functor $F$, and elements $x,y \in G$, that is morphisms in $\mathrm{B}G$, we have $F(x)F(y) = F(xy)$ which is exactly a group homomorphism. ///
  
  % Given two groups $G$ and $H$, a functor $\mathrm{B}G \ra \mathrm{B}H$ takes the object in $\mathrm{B}G$ to the object in $\mathrm{B}H$, the identity morphism to the identity morphism, and each other morphism in $\mathrm{B}G$ to a morhpism in $\mathrm{B}H$. (What is this question asking?)


  \exercise{1.3.iii} Pick a category $C$ where $\obj C = \{a,b,b',c\}$ and $|\obj C| = 4$ such that the only morphisms inlcuded beyond the identities are $f : a \ra b$ and $g: b' \ra c$. 

  Now define a functor $F : C \ra D$ for some category $D$ where $\obj D = \{Fa, Fb = Fb', Fc\}$ and $|\obj D| = 3$. Let $F$ preserve identities and $Ff : Fa \ra Fb$ and $Fg : Fb \ra Fc$. Observe that this is in fact a functor because it preserves identities and there are no nontrivial compositions in $C$. 

  Because $Fg \circ Ff$ is a morphism in $D$, but is not in the image of $F$, the image of $F$ is not a sub category. ///


  \exercise{1.3.iv} First observe that $C(c,-)$ and $C(-,c)$ both preserve identities. Consider the action of $C(c,-)$ on $\id_x : x \ra x$. $\id_x$ maps to post composition by the identity on $C(c,x)$ which is the identity morphism for $C(c,x)$. Similar reason applies for $C(-,c)$. 

  Now pick two morphisms $f:x \ra y$ and $g:y \ra z$ in $C$. Then $C(c,-)g \cdot C(c,-)f = g_*f_* = (gf)_* = C(c,-)(gf)$. Hence $C(c,-)$ is in fact a functor. The same holds for $C(-,c)$ by duality. ///

  % \exercise{1.3.v} I'm not sure I understand the question but I think the pairs of functors are duals: A functor $F : C^{\op} \ra D$ sends a morphism $f : x \ra y$ to $Ff : Fy \ra Fx$. We may construct a new functor $F' : C \ra D^{\op}$ such that $f : y \ra x \mapsto F'f : F'x \ra F'y$.

  \exercise{1.3.viii} Consider the trivial Functor $\mathbf{0} : \mathrm{Set} \ra \mathrm{Set}$ that maps every set to $\emptyset$ and every morphism to $\id_\emptyset$. Observe that this is indeed a functor because $\mathbf{0}(\id_x) = \id_{\mathbf 0 x} = \id_\emptyset$ and $\mathbf 0 (gf) = \mathbf 0 (g) \mathbf 0 (f) = \id_\emptyset$. Now pick any morphism in Set that is not an isomorphism. It is mapped to $\id_\emptyset$ which is an isomorphism. Hence functors do not reflect isomorphisms. ///

  \exercise{1.3.vi} Consider the map $\mathrm{dom} : F \commaArrow G \ra D$ that takes an object $(d,e,f) \mapsto d$ and a morphism $(h,k) \mapsto h$. Now pick an identity morphism $(\id_d, \id_e)$; clearly this functor preserves identity. Now pick a composable pair of morphisms $(h,k)$ and $(h',k')$. We have $\mathrm{dom} ((h',k')(h,k))= \mathrm{dom}(h'h,k'k) = h'h = \mathrm{dom}(h',k')\mathrm{dom}(h,k)$. Hence $\mathrm{dom}$ is a functor. Define $\mathrm{cod}$ analagously. ///


  \exercise{1.3.x} Let $\mathrm{cl}(g)$ be the conjugacy class of $g$ and $\mathrm{Cl}(G)$ be the set of all conjugacy classes of $G$ (I don't know what the standard notation for this is). Consider the following mapping:
  \begin{align*}
    \mathrm{Conj} : \mathrm{Group} &\longrightarrow \mathrm{Set} \\
    G &\longmapsto \text{Cl}(G) \\
    (f : G \ra H) &\longmapsto (f': \mathrm{Cl}(G) \ra \mathrm{Cl}(H))
  \end{align*}
  Where $f'$ maps $\mathrm{cl}(x)$ to $\mathrm{cl}(f(x))$. Note that this is indeed a well defined map: pick an element $b$ in $G$. Then $f'(b) = \mathrm{cl}(f(b))$. For any other element $a \in \mathrm{cl}(b)$, we have $f'(a) = \mathrm{cl}(f(a)) = \mathrm{cl}(f(g)f(b)f(g)^{-1}) = \mathrm{cl}(f(b))$. 

  Observe that $\oper{Conj}$ preserves identities. Now pick $h : G \ra H$ and $k : H \ra I$. Then $\oper{Conj}(kh) = f' : \oper{Cl}(G) \ra \oper{Cl}(I) = \oper{Conj}(k)\oper{Conj}(h)$. Hence $\oper{Conj}$ is a functor. 

  Now pick a pair of groups such that the cardinalities of their sets of conjugacy classes differ. If there was an isomorphism between the two groups, say $k : G \ra H$, then $\mathrm{Conj} k : \mathrm{Cl}(G) \ra \mathrm{Cl}(H)$ would be an isomorphism as well. But this cannot be because there are no isomorphisms between sets with different cardinalities. ///


  \subsection*{Section 1.4}

  \exercise{1.4.i} Consider the inverse of a component of the natural isomorphism $\alpha : F \Rightarrow G$ such as $\alpha^{-1}_c : Gc \ra Fc$. It remains to show that the following diagram commutes: 
  \begin{center}
    \begin{tikzcd}
      Gc \arrow[d, "G\phi"'] \arrow[r, "\alpha_c^{-1}"] & Fc \arrow[d, "F\phi"] \\
      Gc' \arrow[r, "\alpha_{c'}^{-1}"]& Fc'
    \end{tikzcd}
  \end{center}

  By the naturality of $\alpha$ we have that $G\phi \cdot \alpha_c = \alpha_{c'} \cdot F\phi$. Hence by composing each side with $\alpha_c^{-1}$ and $\alpha_{c'}^{-1}$ we get $\alpha_{c'}^{-1} \cdot G\phi = F\phi \cdot \alpha_{c}^{-1}$. Hence $\alpha^{-1} : G \Rightarrow F$ is a natural isomorphism. ///


  \exercise{1.4.ii} Pick two groups $H, K$ and two functors $F, G : \text{B}H \rightrightarrows \text{B}K$ such that there exists a natural transformation $\alpha : F \Rightarrow G$. By exercise 1.3.i, $F$ and $G$ are group homomorphisms. Now note that because each category contains only one element, there is exactly one component of $\alpha$, say $k$. Now pick a morphism (group element) in B$H$, say $h$ and consider the corresponding naturality square. Examining it shows that $kF(h) = G(h)k$ so $F(h) = k^{-1}G(h)k$. Hence we may say that $\alpha$ is an element $k \in K$ such that for every $h \in H$, $F(h) = k^{-1}G(h)k$. ///

  \exercise{1.4.iv} Observe that $f$ and $g$ define the same natural transformation exactly when their components are equal. The naturality condition for $f_*$ is that $h^*f_* = f_*h^*$ and for $g_*$ it is $h^*g_* = f_*g^*$. Hence because $f$ and $g$ are distinct morphisms, we do not have $h^*f_* = h^*g_*$, for example if $h^*$ is the identity. ///

  \exercise{1.4.vi} The components of a natural transformation are defined to be morphisms in the shared target category of two functors. Hence the definition of natural (or extranatural) transformation does not allow for different target categories because the same notion of morphism does not apply. ///

  % \hl{Questions:} Why is $A$ included in the definition extrantural transformation? Why not just $B$ and $B^\op$? \\
  % Are there notions of morphism between objects in different categories? (Not functors, but a morphism from an object $x \in C$ to an object $y \in D$).

  \subsection*{Section 1.5}
  \exercise{1.5.i} Let $\phi : 0 \ra 1$ be the single nonidentity morphism in $\mathbf{2}$. Then consider the category $C \times \mathbf{2}$ and pick a morphism $f : x \ra y$ in $C$. We have
  \begin{center}
  \begin{tikzcd}
    (x,0) \arrow[r, "(\id_x{,}\phi)"] \arrow[d, "(f{,}\id_0)"'] \arrow[dr, "(f{,}\phi)" description] & (x,1) \arrow[d, "(f{,}\id_1)"] \\
    (y,0) \arrow[r, "(\id_y{,}\phi)"'] & (y,1)
  \end{tikzcd} 
  \end{center}
  A quick check shows that this diagram commutes. Now apply $H$ to $C \times \mathbf{2}$. Observe that because $F = Hi_0$, we have $H(f,\id_0) = Ff$. Similarly for $G$, we have $H(f,\id_1) = Gf$. Now because $H$ is itself a single functor, it preserves compositions of morphisms. Hence the above square is lifted to $D$ where it forms a naturality square for $f$ for some natural transformation where $H(\id_x, \phi)$ and $H(\id_y,\phi)$ are the components. 
  
  Let $\mathcal{H}$ be the set of all functors $ C \times \mathbf{2} \ra D$ that restrict along $i_0$ and $i_1$ to $F$ and $G$ and $\mathcal{A}$ be the set of natural transformations from $F$ to $G$. Observe then that applying $H$ defines a function $\mathcal{F} : \mathcal{H} \ra \mathcal{A}$.

  Further, this mapping is injective because for any $H,K \in \mathcal{H}$, $\mathcal{F}(H) = \mathcal{F}(K)$ implies that all the components are the same


  \exercise{1.5.iii} Let the two ismorphisms be $g : a \ra a'$ and $h : b \ra b'$ and let the leftmost diagram define $f'$. Then $f' = hfg^{-1}$. Hence the second diagram commutes because $f'g = hf$, the third because $h^{-1}f' = fg^{-1}$, and the fourth because $h^{-1}f'g = f$. ///


  \exercise{1.5.iv}


  \exercise{1.5.ix} Theorem 1.5.9 says that a functor defining an equivalence of categories is fully faithful. Hence, for a category to be equivalent to a locally small category, the hom sets of each category must be isomorphic. Sets are only isomorphic to other sets; therefore the equivalent category is also locally small. ///

  \exercise{1.5.xi}
  Consider the inclusion $\text{Ab} \ra \text{Group}$. It is fully faithful because the hom sets between abelian groups are the same in each category. It is not essentially surjective because nonabelian groups are not isomorphic to abelian groups. 

  Consider $\text{Field} \ra \text{Ring}$.
\end{document}