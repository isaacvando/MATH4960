\documentclass[12pt]{article}
\usepackage{amsmath}
\usepackage{amsfonts}
\usepackage{amssymb}
\usepackage{mathtools}
\usepackage{parskip}
\usepackage{fancyhdr}
\usepackage{geometry}
\geometry{
  a4paper,
  left= 1in,
  right= 1in,
  top=1.42in,
}
\pagestyle{fancy}
\renewcommand{\headrulewidth}{0.5pt}
\renewcommand{\footrulewidth}{0.5pt}
\newcommand{\N}{\mathbb{N}}
\newcommand{\Z}{\mathbb{Z}}
\newcommand{\Q}{\mathbb{Q}}
\newcommand{\R}{\mathbb{R}}
\newcommand{\C}{\mathbb{C}}
\newcommand{\F}{\mathbb{F}}
\newcommand{\ER}{\mathcal{R}}
\newcommand{\im}{\text{im}}
\newcommand{\id}{\text{id}}
\renewcommand{\char}{\text{char}}
\newcommand{\coker}{\text{coker}}
\newcommand{\rank}{\text{rank}}
\newcommand{\row}{\text{row}}
\newcommand{\col}{\text{col}}
\newcommand{\nul}{\text{nul}}
\newcommand{\Tr}{\text{Tr}}
\newcommand{\exercise}[1]{\fbox{\textbf{Exercise #1}}}
\newcommand{\bitem}[1]{\item[\textbf{#1}]}
\newcommand{\nxn}{M_{n \times n}(K)}
\newcommand{\adj}{\text{adj}}
\newcommand{\Mult}{\text{Mult}_K(V^m, W)}
\newcommand{\Alt}{\text{Alt}_K(V^m, W)}
\newcommand{\ra}{\rightarrow}
\newcommand{\sgn}{\text{sgn}}
\newcommand{\amult}{\text{a-Mult}}
\newcommand{\gmult}{\text{g-Mult}}
\renewcommand{\null}{\text{null}}
\newcommand{\Min}{\text{Min}}
\newcommand{\End}{\text{End}}
\newcommand{\tors}{M_{\text{tors}}}
\newcommand{\ann}[2][R]{\text{Ann}_{#1}(#2)}
\newcommand{\kx}{K[x]}

\newcommand\restr[2]{{% we make the whole thing an ordinary symbol
\left.\kern-\nulldelimiterspace % automatically resize the bar with \right
#1 % the function
% \littletaller % pretend it's a little taller at normal size
\right|_{#2} % this is the delimiter
}}
\newcommand{\littletaller}{\mathchoice{\vphantom{\big|}}{}{}{}}

\lhead{Isaac Van Doren}
\chead{Chapter 1}
\rhead{\today}
\lfoot{}
\rfoot{Category Theory}

\begin{document}

  \subsection*{Section 1}
  \exercise{1} 
  \begin{itemize}
    \item[(i)] Let $f : x \ra y$ be a morphism and $g, g' : y \ra x$ be inverses. Then $fg = \id_y$ and $gf = \id_x$ (resp $g'$). Hence $gf = g'f$ so 
    \begin{align*} 
      gfg &= g'fg \\
      g &= g'
    \end{align*}
  Therefore any morphism has at most one inverse. ///

  \item[(ii)] Suppose there are morphisms $g,h : y \rightrightarrows x$ such that $fg = \id_y$ and $hf = \id_x$. Then
    \begin{align*}
      hfg &= h \\
      g &= h
    \end{align*}
  Hence $fg = \id_y$ and $gf = \id_x$ so $f$ is an isomorphism. ///
  \end{itemize}

  \exercise{2}
  Consider a category $C$ and the subset of isomorphisms it contains. The subset contains the identity morphisms for each object because the identity is an isomorphism. The subset inherits associativity from $C$. Now pick two morphisms $f : x \ra y$ and $g : y \ra z$ and consider $gf$. Because $f$ and $g$ are isomorphisms, $f^{-1}$ and $g^{-1}$ exist and are isomorphisms so are also in the set. Hence 

\end{document}