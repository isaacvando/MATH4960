\documentclass[12pt]{article}
\usepackage{amsmath}
\usepackage{amsfonts}
\usepackage{amssymb}
\usepackage{mathtools}
\usepackage{parskip}
\usepackage{fancyhdr}
\usepackage{tikz-cd}
\usepackage{geometry}
\usepackage[normalem]{ulem}
\newcommand\hl{\bgroup\markoverwith
    {\textcolor{yellow}{\rule[-.5ex]{.1pt}{2.5ex}}}\ULon}

\geometry{
  a4paper,
  left= 1in,
  right= 1in,
  top=1.42in,
}
\pagestyle{fancy}

\renewcommand{\headrulewidth}{0.5pt}
\renewcommand{\footrulewidth}{0.5pt}
\newcommand{\N}{\mathbb{N}}
\newcommand{\Z}{\mathbb{Z}}
\newcommand{\Q}{\mathbb{Q}}
\newcommand{\R}{\mathbb{R}}
\newcommand{\C}{\mathbb{C}}
\newcommand{\F}{\mathbb{F}}
\newcommand{\ER}{\mathcal{R}}
\newcommand{\im}{\text{im}}
\newcommand{\id}{\text{id}}
\renewcommand{\char}{\text{char}}
\newcommand{\coker}{\text{coker}}
\newcommand{\rank}{\text{rank}}
\newcommand{\row}{\text{row}}
\newcommand{\col}{\text{col}}
\newcommand{\nul}{\text{nul}}
\newcommand{\Tr}{\text{Tr}}
\newcommand{\exercise}[1]{\fbox{\textbf{Exercise #1}}}
\newcommand{\bitem}[1]{\item[\textbf{#1}]}
\newcommand{\nxn}{M_{n \times n}(K)}
\newcommand{\adj}{\text{adj}}
\newcommand{\Mult}{\text{Mult}_K(V^m, W)}
\newcommand{\Alt}{\text{Alt}_K(V^m, W)}
\newcommand{\ra}{\rightarrow}
\newcommand{\sgn}{\text{sgn}}
\newcommand{\amult}{\text{a-Mult}}
\newcommand{\gmult}{\text{g-Mult}}
\renewcommand{\null}{\text{null}}
\newcommand{\Min}{\text{Min}}
\newcommand{\End}{\text{End}}
\newcommand{\tors}{M_{\text{tors}}}
\newcommand{\ann}[2][R]{\text{Ann}_{#1}(#2)}
\newcommand{\kx}{K[x]}
\newcommand{\cat}[1]{\upshape{\sffamily{#1}}}
\newcommand{\commaArrow}{\!\downarrow\!}
\newcommand{\oper}{\operatorname}
\newcommand{\op}{\mathrm{op}}
\newcommand{\obj}{\mathrm{obj}}
\newcommand{\mor}{\mathrm{mor}}

\newcommand\restr[2]{{% we make the whole thing an ordinary symbol
\left.\kern-\nulldelimiterspace % automatically resize the bar with \right
#1 % the function
% \littletaller % pretend it's a little taller at normal size
\right|_{#2} % this is the delimiter
}}
\newcommand{\littletaller}{\mathchoice{\vphantom{\big|}}{}{}{}}

\lhead{Isaac Van Doren}
\chead{Section 5}
\rhead{\today}
\lfoot{}
\rfoot{Category Theory}

\begin{document}
  % \subsection*{Section 1.5}
  \exercise{1.5.i} Let $\phi : 0 \ra 1$ be the single nonidentity morphism in $\mathbf{2}$. Then consider the category $C \times \mathbf{2}$ and pick a morphism $f : x \ra y$ in $C$. We have
  \begin{center}
  \begin{tikzcd}
    (x,0) \arrow[r, "(\id_x{,}\phi)"] \arrow[d, "(f{,}\id_0)"'] \arrow[dr, "(f{,}\phi)" description] & (x,1) \arrow[d, "(f{,}\id_1)"] \\
    (y,0) \arrow[r, "(\id_y{,}\phi)"'] & (y,1)
  \end{tikzcd} 
  \end{center}
  A quick check shows that this diagram commutes. Now apply $H$ to $C \times \mathbf{2}$. Observe that because $F = Hi_0$, we have $H(f,\id_0) = Ff$. Similarly for $G$, we have $H(f,\id_1) = Gf$. Now because $H$ is itself a single functor, it preserves compositions of morphisms. Hence the above square is lifted to $D$ where it forms a naturality square for $f$ for some natural transformation where $H(\id_x, \phi)$ and $H(\id_y,\phi)$ are the components. 
  
  Let $\mathcal{H}$ be the set of all functors $ C \times \mathbf{2} \ra D$ that restrict along $i_0$ and $i_1$ to $F$ and $G$ and $\mathcal{A}$ be the set of natural transformations from $F$ to $G$. Observe then that applying $H$ defines a function $\mathcal{F} : \mathcal{H} \ra \mathcal{A}$.

  We may now define $\mathcal{F}^{-1}$. For any $\alpha : F \Rightarrow G$, map the component $\alpha_x$ to $(\id_x,\phi)$. This is enough to define a particular functor in $\mathcal{H}$ as the other mappings are constrained by $F$ and $G$. 

  Hence $\mathcal{H} \cong \mathcal{A}$. ///


  \exercise{1.5.iii} Let the two ismorphisms be $g : a \ra a'$ and $h : b \ra b'$ and let the leftmost diagram define $f'$. Then $f' = hfg^{-1}$. Hence the second diagram commutes because $f'g = hf$, the third because $h^{-1}f' = fg^{-1}$, and the fourth because $h^{-1}f'g = f$. ///


  \exercise{1.5.iv}
  \begin{itemize}
    \item[(i)] Because $F$ is fully faithful, $C(y,x) \cong D(Fy,Fx)$. Hence $(Ff)^{-1}$ has a preimage in $C$, say $g$. Thus $Ff \circ Fg = \id_y = F(fg) =  F\id_y$. Hence $fg = \id_y$. Similar reasoning shows that $gf = \id_x$. Therefore $f$ is an isomorphism. ///
    
    \item[(ii)] Because $F$ is fully faithful, we know that for any $x,y \in C$, $C(x,y) \cong D(Fx,Fy)$. There is at least one morphism in $D(Fx,Fy)$, namely the isomorphism $Ff$. Thus there is a morphism $f \in C(x,y)$. Now apply (i) to see that $f$ is an isomorphism so $x \cong y$. ///
    
    % $\phi : Fx \ra Fy$. Call the preimage of that morphism $f$ in $C$. By similar reasoning, there is a morphism $g \in C$ that is the preimage of $\phi^{-1}$. $\phi \circ \phi^{-1} = Ff\circ Fg = \id_{F_y} = F(fg) = F\id_y$ by functoriality. Hence $fg = \id_y$. Similar reasoning shows that $gf = \id_x$. Therefore $x \cong y$. ///
  \end{itemize}


  \exercise{1.5.ix} Theorem 1.5.9 says that a functor defining an equivalence of categories is fully faithful. Hence, for a category to be equivalent to a locally small category, the hom sets of each category must be isomorphic. Sets are only isomorphic to other sets; therefore the equivalent category is also locally small. ///

  \exercise{1.5.xi}
  Consider the inclusion $\text{Ab} \ra \text{Group}$. It is fully faithful because the hom sets between abelian groups are the same in each category. It is not essentially surjective because nonabelian groups are not isomorphic to abelian groups. ///

  % Consider $\text{Field} \ra \text{Ring}$.
\end{document}