\documentclass[12pt]{article}
\usepackage{amsmath}
\usepackage{amsfonts}
\usepackage{amssymb}
\usepackage{mathtools}
\usepackage{parskip}
\usepackage{fancyhdr}
\usepackage{tikz-cd}
\usepackage{geometry}
\usepackage[normalem]{ulem}
\newcommand\hl{\bgroup\markoverwith
    {\textcolor{yellow}{\rule[-.5ex]{.1pt}{2.5ex}}}\ULon}

\geometry{
  a4paper,
  left= 1in,
  right= 1in,
  top=1.42in,
}
\pagestyle{fancy}

\renewcommand{\headrulewidth}{0.5pt}
\renewcommand{\footrulewidth}{0.5pt}
\newcommand{\N}{\mathbb{N}}
\newcommand{\Z}{\mathbb{Z}}
\newcommand{\Q}{\mathbb{Q}}
\newcommand{\R}{\mathbb{R}}
\newcommand{\C}{\mathbb{C}}
\newcommand{\F}{\mathbb{F}}
\newcommand{\ER}{\mathcal{R}}
\newcommand{\im}{\text{im}}
\newcommand{\id}{\text{id}}
\renewcommand{\char}{\text{char}}
\newcommand{\coker}{\text{coker}}
\newcommand{\rank}{\text{rank}}
\newcommand{\row}{\text{row}}
\newcommand{\col}{\text{col}}
\newcommand{\nul}{\text{nul}}
\newcommand{\Tr}{\text{Tr}}
\newcommand{\exercise}[1]{\fbox{\textbf{Exercise #1}}}
\newcommand{\bitem}[1]{\item[\textbf{#1}]}
\newcommand{\nxn}{M_{n \times n}(K)}
\newcommand{\adj}{\text{adj}}
\newcommand{\Mult}{\text{Mult}_K(V^m, W)}
\newcommand{\Alt}{\text{Alt}_K(V^m, W)}
\newcommand{\ra}{\rightarrow}
\newcommand{\sgn}{\text{sgn}}
\newcommand{\amult}{\text{a-Mult}}
\newcommand{\gmult}{\text{g-Mult}}
\renewcommand{\null}{\text{null}}
\newcommand{\Min}{\text{Min}}
\newcommand{\End}{\text{End}}
\newcommand{\tors}{M_{\text{tors}}}
\newcommand{\ann}[2][R]{\text{Ann}_{#1}(#2)}
\newcommand{\kx}{K[x]}
\newcommand{\cat}[1]{\upshape{\sffamily{#1}}}
\newcommand{\commaArrow}{\!\downarrow\!}
\newcommand{\oper}{\operatorname}
\newcommand{\op}{\mathrm{op}}
\newcommand{\obj}{\mathrm{obj}}
\newcommand{\mor}{\mathrm{mor}}

\newcommand\restr[2]{{% we make the whole thing an ordinary symbol
\left.\kern-\nulldelimiterspace % automatically resize the bar with \right
#1 % the function
% \littletaller % pretend it's a little taller at normal size
\right|_{#2} % this is the delimiter
}}
\newcommand{\littletaller}{\mathchoice{\vphantom{\big|}}{}{}{}}

\lhead{Isaac Van Doren}
\chead{Chapter 1}
\rhead{\today}
\lfoot{}
\rfoot{Category Theory}

\begin{document}
  \subsection*{Section 6}
  \exercise{1.6.i} Let $i$ and $t$ be initial and terminal in $C$ respectively and suppose there is a morphism $f : t \ra i$. Because $i$ and $t$ are terminal and initial, there is a unique morphism $g : i \ra t$. Now observe that $fg = \id_i$ and $gf = \id_t$ because there is a unique morphism $i \ra i$ and a unique morphism $t \ra t$ which must then each be the identity. Thus $t \cong i$. ///

  \exercise{1.6.ii}
  Let $p$ and $q$ be terminal objects. Then there are unique morphism $p \ra q$ and $q \ra p$. Because the only morphisms $p \ra p$ and $q \ra q$ are the identities, $pq$ and $qp$ are the identities so $p \cong q$ uniquely. ///

  \exercise{1.6.iv} The forgetful functor \textbf{Ring} $\ra$ \textbf{Set} is faithful but does not preserve epimorphisms because the inclusion $\Z \hookrightarrow \Q$ is epic but not a surjection. 

  Argue by duality...

  \exercise{1.6.v} Consider the concrete category $(\mathbf{2},U)$ where $\mathbf{2}$ is the category containing a single morphism $0 \ra 1$ and where $U : \mathbf{2} \ra \textbf{Set}$ takes 0 to a singleton set, 1 to a doubleton set, and the morphism to some morphism $U0 \ra U1$. Observe that the morphism in \textbf{2} is a monomorphism but its underlying function is not injective because there are no injections between finite sets of different cardinalities.
  
  As we saw in a previous exercise, the inclusion $\Z \hookrightarrow \Q$ in the category of Rings is an epimorphism but the underlying function is not a surjection. ///

  % \exercise{1.6.vi}
\end{document}