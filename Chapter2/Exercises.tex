\documentclass[12pt]{article}
\usepackage{amsmath}
\usepackage{amsfonts}
\usepackage{amssymb}
\usepackage{mathtools}
\usepackage{parskip}
\usepackage{fancyhdr}
\usepackage{tikz-cd}
\usepackage{geometry}
\usepackage[normalem]{ulem}
\newcommand\hl{\bgroup\markoverwith
    {\textcolor{yellow}{\rule[-.5ex]{.1pt}{2.5ex}}}\ULon}

\geometry{
  a4paper,
  left= 1in,
  right= 1in,
  top=1.42in,
}
\pagestyle{fancy}

\renewcommand{\headrulewidth}{0.5pt}
\renewcommand{\footrulewidth}{0.5pt}
\newcommand{\N}{\mathbb{N}}
\newcommand{\Z}{\mathbb{Z}}
\newcommand{\Q}{\mathbb{Q}}
\newcommand{\R}{\mathbb{R}}
\newcommand{\C}{\mathbb{C}}
\newcommand{\F}{\mathbb{F}}
\newcommand{\ER}{\mathcal{R}}
\newcommand{\im}{\text{im}}
\newcommand{\id}{\text{id}}
\renewcommand{\char}{\text{char}}
\newcommand{\coker}{\text{coker}}
\newcommand{\rank}{\text{rank}}
\newcommand{\row}{\text{row}}
\newcommand{\col}{\text{col}}
\newcommand{\nul}{\text{nul}}
\newcommand{\Tr}{\text{Tr}}
\newcommand{\exercise}[1]{\fbox{\textbf{Exercise #1}}}
\newcommand{\bitem}[1]{\item[\textbf{#1}]}
\newcommand{\nxn}{M_{n \times n}(K)}
\newcommand{\adj}{\text{adj}}
\newcommand{\Mult}{\text{Mult}_K(V^m, W)}
\newcommand{\Alt}{\text{Alt}_K(V^m, W)}
\newcommand{\ra}{\rightarrow}
\newcommand{\sgn}{\text{sgn}}
\newcommand{\amult}{\text{a-Mult}}
\newcommand{\gmult}{\text{g-Mult}}
\renewcommand{\null}{\text{null}}
\newcommand{\Min}{\text{Min}}
\newcommand{\End}{\text{End}}
\newcommand{\tors}{M_{\text{tors}}}
\newcommand{\ann}[2][R]{\text{Ann}_{#1}(#2)}
\newcommand{\kx}{K[x]}
\newcommand{\cat}[1]{\upshape{\sffamily{#1}}}
\newcommand{\commaArrow}{\!\downarrow\!}
\newcommand{\oper}{\operatorname}
\newcommand{\op}{\mathrm{op}}
\newcommand{\obj}{\mathrm{obj}}
\newcommand{\mor}{\mathrm{mor}}
\renewcommand{\P}{\mathcal{P}}

\newcommand\restr[2]{{% we make the whole thing an ordinary symbol
\left.\kern-\nulldelimiterspace % automatically resize the bar with \right
#1 % the function
% \littletaller % pretend it's a little taller at normal size
\right|_{#2} % this is the delimiter
}}
\newcommand{\littletaller}{\mathchoice{\vphantom{\big|}}{}{}{}}

\lhead{Isaac Van Doren}
\chead{Chapter 2}
\rhead{\today}
\lfoot{}
\rfoot{Category Theory}

\begin{document}
  \subsection*{Section 1}
  \exercise{2.1.i} 


  \exercise{2.1.ii} If $F$ is representable then we have a natural isomorphism $\alpha$ and the following commutative square:
  \begin{center}
    \begin{tikzcd}
        Fx \arrow[r, "\alpha_x", leftrightarrow] \arrow[d, "Ff"'] & C(c,x) \arrow[d, "f_*"] \\
        Fy \arrow[r, "\alpha_y", leftrightarrow] & C(c,y)
    \end{tikzcd}
  \end{center}

  Now suppose $f$ is monic and pick two morphisms $g,h : W \rightrightarrows C(c,x)$ for some set $W$, such that $f_* g = f_* h$. This is equivalent to saying $\forall w \in W, f_*(g(w)) = f_*(h(w)) \implies f \circ (g(w)) = f \circ (h(w)) \implies g(w) = h(w) \implies g = h$ by the fact that $f$ is monic. Hence $f_*$ is monic. 

  Now, by the square, we have $\alpha_y Ff = f_* \alpha_x \implies Ff = \alpha_y^{-1} f_* \alpha_x$. Hence $Ff$ is a composition of three injective maps which together are also injective. Therefore $F$ preserves monomorphisms. ///
 
\end{document}