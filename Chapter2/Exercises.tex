\documentclass[12pt]{article}
\usepackage{amsmath}
\usepackage{amsfonts}
\usepackage{amssymb}
\usepackage{mathtools}
\usepackage{parskip}
\usepackage{fancyhdr}
\usepackage{tikz-cd}
\usepackage{geometry}
\usepackage[normalem]{ulem}
\newcommand\hl{\bgroup\markoverwith
    {\textcolor{yellow}{\rule[-.5ex]{.1pt}{2.5ex}}}\ULon}

\geometry{
  a4paper,
  left= 1in,
  right= 1in,
  top=1.42in,
}
\pagestyle{fancy}

\renewcommand{\headrulewidth}{0.5pt}
\renewcommand{\footrulewidth}{0.5pt}
\newcommand{\N}{\mathbb{N}}
\newcommand{\Z}{\mathbb{Z}}
\newcommand{\Q}{\mathbb{Q}}
\newcommand{\R}{\mathbb{R}}
\newcommand{\C}{\mathbb{C}}
\newcommand{\F}{\mathbb{F}}
\newcommand{\ER}{\mathcal{R}}
\newcommand{\im}{\text{im}}
\newcommand{\id}{\text{id}}
\renewcommand{\char}{\text{char}}
\newcommand{\coker}{\text{coker}}
\newcommand{\rank}{\text{rank}}
\newcommand{\row}{\text{row}}
\newcommand{\col}{\text{col}}
\newcommand{\nul}{\text{nul}}
\newcommand{\Tr}{\text{Tr}}
\newcommand{\exercise}[1]{\fbox{\textbf{Exercise #1}}}
\newcommand{\bitem}[1]{\item[\textbf{#1}]}
\newcommand{\nxn}{M_{n \times n}(K)}
\newcommand{\adj}{\text{adj}}
\newcommand{\Mult}{\text{Mult}_K(V^m, W)}
\newcommand{\Alt}{\text{Alt}_K(V^m, W)}
\newcommand{\ra}{\rightarrow}
\newcommand{\sgn}{\text{sgn}}
\newcommand{\amult}{\text{a-Mult}}
\newcommand{\gmult}{\text{g-Mult}}
\renewcommand{\null}{\text{null}}
\newcommand{\Min}{\text{Min}}
\newcommand{\End}{\text{End}}
\newcommand{\tors}{M_{\text{tors}}}
\newcommand{\ann}[2][R]{\text{Ann}_{#1}(#2)}
\newcommand{\kx}{K[x]}
\newcommand{\cat}[1]{\upshape{\sffamily{#1}}}
\newcommand{\commaArrow}{\!\downarrow\!}
\newcommand{\oper}{\operatorname}
\newcommand{\op}{\mathrm{op}}
\newcommand{\obj}{\mathrm{obj}}
\newcommand{\mor}{\mathrm{mor}}
\renewcommand{\P}{\mathcal{P}}

\newcommand\restr[2]{{% we make the whole thing an ordinary symbol
\left.\kern-\nulldelimiterspace % automatically resize the bar with \right
#1 % the function
% \littletaller % pretend it's a little taller at normal size
\right|_{#2} % this is the delimiter
}}
\newcommand{\littletaller}{\mathchoice{\vphantom{\big|}}{}{}{}}

\lhead{Isaac Van Doren}
\chead{Chapter 2}
\rhead{\today}
\lfoot{}
\rfoot{Category Theory}

\begin{document}
  \subsection*{Section 1}
  \exercise{2.1.i} The question asks us to the consider natural transformations between $\mathbf{Cat}(\mathbf{1},-)$ and $\mathbf{Cat}(\mathbf{2},-)$ induced by functors between $\mathbf{1}$ and $\mathbf{2}$.

  For the collapsing functor $! : \mathbf{2} \ra \mathbf{1}$, we have the following naturality square:
  \begin{center}
    \begin{tikzcd}
        \mathbf{Cat}(\mathbf{1},x) \arrow[r, "!^*"] \arrow[d, "f_*"] & \mathbf{Cat}(\mathbf{2},x) \arrow[d, "f_*"] \\
        \mathbf{Cat}(\mathbf{1},y) \arrow[r, "!^*"] & \mathbf{Cat}(\mathbf{2},y)
    \end{tikzcd}
  \end{center}
  This diagram commutes because for any suitable input $x$, $f_*!^*(x) = !xf = !^*f_*(x)$. 

  We achieve similar results for $0$ and $2$ by drawing a similar diagram for natural transformations from $\mathbf{Cat}(\mathbf{2},-) \ra \mathbf{Cat}(\mathbf{1},-)$. 

  \hl{what else should be said about these natural transformations?}


  \exercise{2.1.ii} If $F$ is representable then we have a natural isomorphism $\alpha$ and the following commutative square:
  \begin{center}
    \begin{tikzcd}
        Fx \arrow[r, "\alpha_x", leftrightarrow] \arrow[d, "Ff"'] & C(c,x) \arrow[d, "f_*"] \\
        Fy \arrow[r, "\alpha_y", leftrightarrow] & C(c,y)
    \end{tikzcd}
  \end{center}

  Now suppose $f$ is monic and pick two morphisms $g,h : W \rightrightarrows C(c,x)$ for some set $W$, such that $f_* g = f_* h$. This is equivalent to saying $\forall w \in W, f_*(g(w)) = f_*(h(w)) \implies f \circ (g(w)) = f \circ (h(w)) \implies g(w) = h(w) \implies g = h$ by the fact that $f$ is monic. Hence $f_*$ is monic. 

  Now, by the square, we have $\alpha_y Ff = f_* \alpha_x \implies Ff = \alpha_y^{-1} f_* \alpha_x$. Hence $Ff$ is a composition of three injective maps which together are also injective. Therefore $F$ preserves monomorphisms.

  
  As for part two, pick a functor $\mathbf{2} \ra \mathbf{Set}$ that takes the morphism to a non-injective function. By the contrapositive of the above statement, the functor is not representable because it does not preserve monomorphisms. ///
 

  \subsection*{Section 2}
  \exercise{2.2.i} To reach the dual of the Yoneda Lemma, consider the category in question to be $C^{\text{op}}$. Then for a functor $F : C^{\op} \ra \mathbf{Set}$, there is a bijection
    $$ \text{Hom}(C^\op(c, -), F) \cong Fc $$
  for any $c \in C^\op$. Now realize that $C(a,b) = C^\op(b,a)$, so we have 
  $$ \text{Hom}(C(-,c), F) \cong Fc $$
  ///

  \exercise{2.2.ii} It does not dualize to that because that is just not the dual statement. ///

  \exercise{2.2.iii} Pick two objects $c,d \in \omega$ and consider $\omega(c,d)$ and $\mathbf{Set}^{\omega^\op}(yc, yd) = \mathbf{Set}^{\omega^\op}(\omega(-,c), \omega(-,d))$. Because $\omega$ is a poset, there is at most one morphism in $\omega(c,d)$


  \exercise{2.2.iv} Suppose $f : x \ra y$ is an isomorphism. Then $f_* : C(-,x) \ra C(-,y)$ defines a natural transformation one direction and $(f^{-1})_* = f_*^{-1}$ defines the other direction.

  Suppose $f_*$ is a natural isomorphism. Then $f_*$ has an inverse, $f_*^{-1}$. This means that we have $f_* \circ f_*^{-1} = \id_y$ and $f_*^{-1} \circ f_* = \id_x$. So, for suitable morphisms, $h,k$, we have $f_*^{-1}(f_*(h)) = f_*^{-1}(fh) = h$ and $f_*(f_*^{-1}(k)) = ff_*^{-1}(k) = k$

  (INSERT PROOF) hence $f$ is an isomorphism. 

  The same reasoning applies for $f^*$. 


  \subsection*{Section 3}
  
\end{document}